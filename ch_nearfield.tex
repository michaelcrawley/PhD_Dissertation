\chapter{The Pressure Signature of Aeroacoustic Sources}
Reference Ani's work here???

\section{Preprocessing: Filtering the Actuator Self-Noise}
Analysis of the near-field response of the forced jet is not immediately straightforward due to acoustic contamination from the actuators themselves [Kearney-fisher]. 
LAFPAs operate on a joule heating principle - the breakdown of the air between the electrodes and the ensuing flow of current results in intense heating of the air. This rapid, localized thermal perturbation produces a compression wave, which excites the shear layer. 
However, this compression wave is still evident as it travels through the near field. 
Multiple compression waves can clearly be seen in \fig{fig:self-noise}, in which a subsonic rectangular jet is being excited at 20 kHz by four LAFPAs on its lower edge. 
\begin{figure}
	\centering
	\includegraphics{Figures/Samimy2010JFM.jpg}
	\caption{Schlieren image highlighting LAFPA compression waves. Reprinted from Samimy 2010 JFM.}
	\label{fig:self-noise}
\end{figure}

Obviously, this is an undesirable effect, as this actuator self-noise may in some cases obscure the hydrodynamic and acoustic response of the jet.
So, in the present work the near-field pressure signals have been preprocessed using a continuous-wavelet-based filtering algorithm, which has been specifically designed to remove the actuator self-noise while leaving the signature of the jet response unaltered. 
An example of this filtering can be found in \fig{fig:preprocessing:wavelet_filter}, where the raw and preprocessed signals have been plotted for $St_{DF} = 0.02$ at $x/D = 1$, $r/D = 1.20$. 
To aid in visualization, the results for multiple excitation periods has been phase-averaged to produce these waveforms. 
As the actuator self-noise is localized in both time and frequency and can be well predicted, a smoothing algorithm in the wavelet domain was found to be the most effective method for removing the undesirable noise while leaving the response of the jet intact. 
A fourth-order Paul wavelet is employed, due to the similarity of its imaginary component to the phase-averaged response of the jet. 
As a result, the energy of the response of the jet is well defined in the wavelet domain, with the actuator self-noise existing as high-frequency, temporally-localized oscillations superimposed on the field. 
After smoothing in the wavelet domain to remove these oscillations, the signal is transformed back into the physical domain where it undergoes another smoothing operation in order to remove small amplitude, high frequency oscillations which may be introduced by the wavelet-smoothing. 
For consistency hereafter, all results examined within this work have been computed from the filtered, rather than the raw, signals.
\begin{figure}
	\centering
	\includegraphics{Figures/NearField_Preprocessing_Filtering.png}
	\caption{Raw and preprocessed near-field pressure.}
	\label{fig:preprocessing:wavelet_filter}
\end{figure} 

\section{Far-field Response}
\section{Acoustic/Hydrodynamic Decomposition}
Much of the difficulty in identifying the aeroacoustic source terms revolves around the dissimilar range of scales and fluctuation intensities of the turbulent eddies in the shear layer and the resulting radiated noise. 
Outside the jet shear layer, in the irrotational near-field of the jet, strong hydrodynamic pressure fluctuations associated directly with the passage of coherent structures in the shear layer and their resultant weak acoustic radiation coexist [Arndt]. 
Beyond this, in the acoustic far-field, the hydrodynamic signature of the coherent structures is nonexistent owing to their strong exponential decay with radial distance.
It is in the irrotational near-field that much work has focused, in order to improve the aeroacoustic community’s understanding of the link between shear layer turbulence and far-field acoustic radiation. 

Owing to the presence of strong hydrodynamic fluctuations dominating the irrotational pressure field near the noise source regions, identification of pure acoustic waves and their corresponding source events is problematic.
A decomposition of the pressure field into its constitutive hydrodynamic and acoustic components is therefore required. 
By identification and prediction of coherence nulls in the near field, Coiffet \etal [cite] showed that the full irrotational near-field consistent primarily as a linear superposition of its hydrodynamic and acoustic components, which lead subsequent researchers to propose linear filters to extract the individual components from the near-field pressure, with varying degrees of success. 

As discussed by Tinney \& Jordan [cite], in a transonic jet in which the large-scale structures are convecting subsonically with respect to the ambient speed of sound, a demarcation of the hydrodynamic and acoustic energy fields can be observed with phase velocity.
This is because the hydrodynamic pressure fluctuations will be aligned with the jet axis, and travelling subsonically. 
Acoustic pressure fluctuations will impinge on the linear microphone array at oblique angles, and therefore will appear as having either sonic or supersonic phase velocity, based on the source location. 
Therefore, a demarcaction between the hydrodynamic and acoustic energy components should be readily identifiable about the sonic wavenumber, $k_a = \omega / a_\infty$.

An illustration of this can be found in \fig{fig:phase_velocity_map}, where the power spectral density of the irrotational near-field pressure for a single microphone array position has been plotted as a function of normalized frequency and (axial) wavenumber.
The sonic velocity has been identified with a dashed line; energies lying above this line correspond to supersonically traveling waves (and hence, acoustic energy) whereas energies below this line correspond to subsonically convecting waves (hydrodynamic energy).
Note that at high wavenumber and frequencies, two distinct energy lobes become readily apparent.

This phase-velocity separation is the basis for the decomposition method of Tinney \& Jordan [cite], which used a Fourier-based wavenumber-frequency filter in a cold, subsonic jet to separate the near-field pressure into supersonically- and subsonically-convecting waves.
The pressure field is first transformed into Fourier space ($k_x,\omega$), as
\begin{equation}
	\hat{p} \left( k_x , \omega \right) = \iint_{\mathbb{R}^2} p(x,t)e^{-i(\omega t - k_x x)}dxdt
\end{equation}
From the transformed pressure field, the hydrodynamic and acoustic fields can then be reconstructed separately, from
\begin{equation}
	p_c (x,t) = \frac{1}{(2 \pi)^2} \iint_{\mathbb{R}^2} \phi_c (k_x,\omega) \hat{p} (k_x,\omega)e^{i(\omega t - k_x x)}dk_x d\omega .
	\label{eq:fourier_filter}
\end{equation}
The component weight vector, $\phi_c \in [0,1]$, is set based on the measured axial phase velocity, $c = \omega / k_x$ in order to filter out either the supersonic or subsonic portion of the spectra. 
 
Grizzi \& Camussi [cite] took a slightly different approach, which utilized a discrete wavelet transform at individual spatial locations in order to decompose the fields based on an energy cutoff. 
The energy threshold was set iteratively, using analysis of two-point correlations of the acoustic and hydrodynamic components between two microphones, in order to ensure that realistic phase-velocities for the components were met. 
The Empirical Mode Decomposition (EMD) based method of Kuo \etal [cite] dispensed with explicit concerns with the phase velocity of the pressure components and instead used the critical frequency, as defined by Arndt \etal [cite], which demarcates the energy dominance of the acoustic and hydrodynamic components in the near-field spectra.
[Include recent Yonglu paper?]
\begin{figure}
	\centering
	\includegraphics[width=3.25in]{Figures/Phase_Velocity_Map.png}
	\caption{Wavenumber-Frequency spectral energy.}
	\label{fig:phase_velocity_map}
\end{figure}

In the current work, the irrotational near-field pressure is decomposed into its constitutive hydrodynamic and acoustic components based on phase-velocity. 
The current method is similar to that of Tinney \& Jordan [cite] in that an axial array of many microphones is used, though it differs in how it identifies components of different phase-velocity.
Here, the filtering will be performed by a spatio-temporal continuous wavelet transform.

\subsection{The Wavelet Transform}
Fourier analysis is commonly employed in the aeroacoustics community to study fundamental aspects of jet noise due to its simplicity and the great abundance of information it can provide. 
However, there is also a great drawback associated with Fourier analysis: while it analyzes a given signal at a distinct frequency, local information for a given event is spread over all spectral coefficients. 
This is due to the fact that the basis functions used by the Fourier transform oscillate indefinitely. 
For a completely stationary signal this is not an issue, however it has become increasingly clear that the jet noise phenomenon is not a stationary process.
Transient events, such as intermittency or the spatial and temporal modulation of a wavepacket, have been shown to be important in the noise generation process. 

Grossman [cite] introduced the wavelet transform in an effort to overcome some of the shortcomings of the Fourier transform.
Unlike the Fourier transform, the wavelet transform involves a convolution of the signal with a set of basis functions which decay to zero at the bounds.
As a direct result, translation of the basis function in space and/or time is now meaningful. 
The basis functions (often referred to as the analyzing or daughter wavelets) are all derived from a single function, the mother wavelet, which must satisfy certain criteria [Farge],
Most notable of of these criteria is that of admissibility, which in essence requires that the wavelet must be of finite energy. 
In practice, it is also helpful to choose a mother wavelet which is well-localized in both the spatio-temporal domain and the frequency domain. 
For a given mother wavelet, $\psi (\vec{x})$, the daughter wavelets can be constructed as
\begin{equation}
	\psi_d \left( \vec{x};s,\vec{\tau},\theta \right) = s^{-n/2} \psi \left( s^{-1} r_{-\theta} \left(\vec{x}-\vec{\tau} \right) \right) \
	\label{eq:daughter_wavelets}
\end{equation}
where $s$ is the scale factor, $\vec{\tau}$ the translation parameter, and in the case of a multidimensional transform, $r_{-\theta}$ is the rotation vector (which can be neglected for an isotropic mother wavelet). 
The $s^{-n/2}$ factor ensures constant energy across all dilations. 
For a specific scale, translation, and rotation, the wavelet transform then becomes
\begin{equation}
	\tilde{f} \left( s, \vec{\tau}, \theta \right) = \int_{\mathbb{R}^n} f \left( \vec{x} \right) \psi^*_d \left( \vec{x};s,\vec{\tau},\theta \right) d^n \vec{x}.
\end{equation}

Because the basis functions of the wavelet transform are of finite energy, the locality of information in the original signal is preserved in the wavelet coefficients. 
This allows the identification, analysis, and reconstruction of localized events in the original signal, something not possible with the Fourier transform, which spreads temporal/spatial information over all transform coefficients. 
This has enabled previous researchers to perform a range of new analysis techniques to turbulence and acoustic phenomena not possible with the traditional Fourier transform. An excellent review of the development of wavelet analysis as well as applications to turbulence can be found in Farge [cite].

Use of a multidimensional, continuous wavelet transform to extract intermittent events with a specific phase-velocity is not immediately straightforward, due to the global nature of the scale factor. 
A `speed-tuning' parameter, $c$, was introduced to the wavelet transform (now specifically referred to as a \textit{spatio-temporal} wavelet transform) by Antoine \etal [cite], who used it for use in motion tracking and identification in two-dimensional images. 
The definition for the daughter wavelets (\ref{eq:daughter_wavelets}) is modified to
\begin{equation}
	\psi_d \left( \vec{x}, t;s,\vec{x}', t' \right) = s^{-n/2} \psi \left(s^{-1} c^{-1/n} (\vec{x}-\vec{x}'), s^{-1} c^{(n-1)/n} (t-t') \right)
\end{equation} 
where $n$ corresponds to the total number of dimensions (temporal and spatial).

The continuous wavelet transform are isometry [Antoine] and hence is invertible.
The original signal may therefore be recovered from the wavelet coefficients as 
\begin{equation}
	f(\vec{x},t) = \frac{1}{C_\delta} \int_0^\infty \frac{ds}{s^{1 + n/2}} \int_{0}^{\infty} \frac{dc}{c} \tilde{f} (\vec{x},t,s,c).
	\label{eq:wavelet_filter}
\end{equation}

The constant factor $C_\delta$ serves as an energy scaling, and appears because we are reconstructing the signal using a different analyzing wavelet (in this case, a delta function) than the mother wavelet used in the forward transform [Torrence,Farge,Antoine]. 
For a given mother wavelet, this factor can be found from
\begin{equation}
	C_\delta = \frac{1}{(2 \pi)^n} \int_0^\infty \frac{ds}{s^{1 + n/2}} \int_{0}^{\infty} \frac{dc}{c} \int_{\mathbb{R}^n} d \vec{k} \int_{-\infty}^{\infty} d\omega \hat{\psi}_d^*
\end{equation}
where $\hat{\psi}_d$ are the daughter wavelets in Fourier space. 
Since we are interested in decomposing the field into the acoustic and hydrodynamic components, a filtered reconstruction can be done quite easily in the wavelet domain by simply modifying the integration limits in \eq{eq:wavelet_filter} to include only speed-tuning parameters corresponding to the subsonic or supersonic portion of the wavelet spectrum.

In this way, this methodology can be thought of as a simple modification of that proposed by Tinney \& Jordan [cite], replacing the Fourier transform in their method with a spatio-temporal wavelet transform. 
The relationship between the wavelet transform and the Fourier transform can be further elucidated by computing the forward transform in the Fourier domain (with the use of the convolution theorem), inserting this into \eq{eq:wavelet_filter}, and reversing the order of the integration:
\begin{equation}
	f_c (\vec{x},t) = \frac{1}{(2 \pi)^n} \int_{\mathbb{R}^{n-1}} d \vec{k} \int_{-\infty}^{\infty} d\omega \hat{f}(\vec{k},\omega) e^{i(\omega t - \vec{k} \cdot \vec{x})} \frac{1}{C_\delta} \int_0^\infty \frac{ds}{s^{1 + n/2}} \int_{0}^{\infty} \frac{dc}{c} \hat{\psi}_d^* (sc^{1/2}k,sc^{-1/2} \omega)
\end{equation}
\begin{equation}
	f_c (\vec{x},t) = \frac{1}{(2 \pi)^n} \int_{\mathbb{R}^{n-1}} d \vec{k} \int_{-\infty}^{\infty} d\omega \hat{f}(\vec{k},\omega) e^{i(\omega t - \vec{k} \cdot \vec{x})} \phi_c (k,\omega)
	\label{eq:wavelet_filter_simplified}
\end{equation}
The appearance of \eq{eq:wavelet_filter_simplified} is identical to that of \eq{eq:fourier_filter}; the difference lies in how filter $\phi_c$ is defined, either explicitly in the Fourier domain in the case of the Fourier filtering or implicitly by the shape of the chosen mother wavelet in the wavelet transform. 
As numerous other researchers have discussed, this leads to an alternative interpretation of the wavelet transform, that of a series of bandpass filters, the passband envelope, centroid, and width being dictated by the scale, speed, and mother wavelet [Farge, torrence].

In fact, computing the convolutions is much faster in the Fourier domain than in the physical domain, so \eq{eq:wavelet_filter_simplified} is the preferred method for computing the spatio-temporal wavelet filter.
The decompositions were performed along each radial microphone array position individually, using the (1+1) dimensional (space-time) Morlet wavelet as the mother wavelet:
\begin{equation}
	\psi (x,t) = e^{i(k_o x + \omega_0 t)} e^{-(x^2 + t^2)/2}
\end{equation}
which the reader will recognize as simply a plane wave modulated by a Gaussian. 
Though simplicity was a factor in this decision, previous results analyzing phase-averaged waveforms in the far-field found acoustic emissions with a characteristic waveform that share some resemblance to the Morlet wavelet [crawley2014]. 
The base oscillation frequencies, 〖$(k_0,\omega_0)$ were set to (±5,5) (the dual sign for $k_0$ being necessary to recover both forward and backward traveling waves), and $\hat{\psi}(k,0) = 0$ and $\hat{\psi}(0,\omega) = 0$ so as to ensure that the mother wavelet met the admissibility criterion.

As the microphone array is irregularly spaced in the axial direction, the pressure field was interpolated onto a regular grid of spacing $1D$ before computation of the discrete Fourier transform. 
In the current work, the local speed of sound was chosen as the phase-velocity demarcation, as opposed to the ambient speed of sound which had been used by previous researchers [Tinney and Jordan]. 
In our case, the jet under study is subsonic and unheated, meaning that the local speed of sound (~320 m/s) is still greater than the jet velocity (~287 m/s) yet lower than the ambient speed of sound (~346 m/s) and hence is a better choice for this particular application.
\subsection{Validation}
\section{Identifying the Acoustic Source Region}