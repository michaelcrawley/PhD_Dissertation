%  The dissertation abstract can only be 500 words.
It has been well-known within the aeroacoustic community that the dominant noise sources in high-speed turbulent jets are related to the large-scale structures which are generated in the initial shear layer by instabilities and which rapidly grow as they convect downstream.
However, the exact dynamics of these large-scale structures which are relevant to the noise generation process are less clear.
This work represents an attempt to study the dynamics and noise generated by the large-scale structures quantitatively and in high-fidelity in a Mach 0.9 turbulent jet using simultaneous pressure and velocity data acquisition systems alongside plasma-based excitation to produce coherent ring vortices in the shear layer.

In the first phase, the irrotational near-field pressure is decomposed into its constitutive acoustic and hydrodynamic components, and two-point cross-correlations are used between the acoustic near-field and far-field in order to identify the dominant noise source region.
Building upon the work of previous researchers, the decomposition is performed using a spatio-temporal wavelet transform, which was found to be more robust than previous algorithms.
Results indicated that for both individual as well as periodic large-scale structures, the dominant noise source region constitutes the upstream region of the jet, ending just before of the end of the potential core (in a time-averaged sense) in the unexcited jet.

The large-scale structure interactions were then investigated by stochastically-estimating the time-resolved velocity fields from the time-resolved near-field pressure.
For computational efficiency, the ensemble velocity snapshots were first decomposed into orthogonal modes, and the a mapping from the near-field pressure to the expansion coefficients was then produced using a feedforward neural network using backpropagation for learning.
The coherent structures generated by the excitation were then identified and tracked using standard vortex identification routines.
For the impulsively-excited jet, the individual structures quickly rolled up into a coherent structure within two jet diameters and then advected until roughly four jet diameters downstream, at which point it underwent a rapid disintegration.
For the periodically-excited jet, multiple smaller-scale structures are initially produced; these quickly merge into a single large-scale structure which matches the excitation wavelength.
Similar to the impulsively-excited structures, these now large-scale structures advect downstream and undergo a rapid disintegration near the end of the potential core. 

Finally, from Ribner's dilatation-based acoustic analogy the aeroacoustic source terms were computed using the time-resolved velocity field produced by the stochastic estimation.
Interpretation of the results is limited however, due to the number of assumptions and simplifications necessary for the computations, given the realities of the available experimental facilities.
Analysis of the computed source fields identified the coherent structures producing a convected wavepacket-like event, centered on the jet lipline though reaching into the potential core.
For the individual vortex rings, a clear modulation of the spatial extent and amplitude was observed as the vortex began to break down just upstream of the end of the potential core.
This behavior is also present for the periodic train of vortices, however it is obscured by an amplification of the source in the upstream region, corresponding to the pairing location for the multiple smaller-scale structures generated by the excitation. 


