%  The dissertation abstract can only be 500 words.
It has been well-known within the aeroacoustic community that the dominant noise sources in high-speed turbulent jets are related to the large-scale structures which are generated in the initial shear layer by instabilities and rapidly grow, interact, and disintegrate as they convect downstream.
However, the exact dynamics of these large-scale structures which are relevant to the noise generation process are less clear.
This work aims to study the dynamics of, and noise generated by, the large-scale structures in high-fidelity in a Mach 0.9 turbulent jet using simultaneous pressure and velocity data acquisition systems alongside plasma-based excitation to produce either individual or periodic coherent ring vortices in the shear layer.

In the first phase, the irrotational near-field pressure is decomposed into its constitutive acoustic and hydrodynamic components, and two-point cross-correlations are used between the acoustic near-field and far-field in order to identify the dominant noise source region.
Building upon the work of previous researchers, the decomposition is performed using a spatio-temporal wavelet transform, which was developed during the current work and found to be more robust than previous techniques.
Results indicated that for both individual as well as periodic large-scale structures, the dominant noise reaching the far-field at low angles to the jet axis is being generated in the upstream region of the jet, ending just before the end of the potential core (in a time-averaged sense) in the unexcited jet.
This is not to say that no noise is generated outside of this region, just that the most energetic and coherent acoustic radiation is emitted here.

The large-scale structure interactions were then investigated by stochastically-estimating the time-resolved velocity fields from time-resolved near-field pressure traces and non-time-resolved planar velocity snapshots.
For computational efficiency, the ensemble velocity snapshots were first decomposed into orthogonal modes, and a mapping from the near-field pressure to the expansion coefficients was then produced using a feedforward neural network using backpropagation for learning.
The coherent structures generated by the excitation were then identified and tracked using standard vortex identification routines.
When exciting the jet at very low frequencies, an individual structure quickly rolled up into a coherent structure within two jet diameters and then advected until roughly four jet diameters downstream, at which point it underwent a rapid disintegration.
For the periodically-excited jet, multiple smaller-scale structures are initially apparent just downstream of the nozzle exit.
These structures quickly undergo multiple mergings to produce a single large-scale structure with a separation distance that matches the excitation wavelength.
Similar to the impulsively-excited structures, these now large-scale structures advect downstream and undergo a rapid disintegration near the end of the potential core. 

Finally, from Ribner's dilatation-based acoustic analogy the aeroacoustic source terms were computed using the time-resolved velocity field produced by the stochastic estimation.
Interpretation of the results is challenging however, due to the number of assumptions and simplifications necessary for the computations given the limitations of the current experimental capabilities.
Analysis of the computed source fields found that the coherent structures produced a convected wavepacket-like event, centered on the jet lipline though reaching into the potential core.
For the individual vortex rings, a clear modulation of the spatial extent and amplitude was observed as the vortex began to break down just upstream of the end of the potential core.
This behavior is also present for the periodic train of vortices observed at higher excitation frequencies, however it is obscured by an amplification of the source in the upstream region where the multiple smaller-scale structures merge.
As the excitation frequency was increased, and multiple vortex mergings occurred before the end of the potential core, the aeroacoustic source associated with the merging amplified such that it was distinct from the vortex disintegration source.

The results from this work indicate that the disintegration of the coherent ring vortices are the dominant aeroacoustic source mechanism for the Mach 0.9, high Reynolds number jet studied here. 
However, the merging of vortices in the initial shear layer was also identified as a non-trivial noise source mechanism in high-speed, turbulent jets. 
Future work will focus on improving the source localization by utilizing acoustic beamforming techniques to identify the source region from the acoustic near-field, in place of the two-point correlations used in this work.
Additionally, the structure dynamics and noise generation process will be explored in high-order azimuthal modes.