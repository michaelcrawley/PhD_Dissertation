\chapter{Dilatation as the Aeroacoustic Source}
Ribner presented an alternative approach to Lighthill's acoustic analogy which posited fluctuating fluid dilatations as the source of aeroacoustic emission [ribner 1962].
This is a reinterpretation of Lighthill's source, which in subsonic, unheated, turbulent jets consists of fluctuating momentum flux.
The driving factor behind Ribner's analysis is the conceptual simplification of the aeroacoustic sources: Lighthill's quadrupoles are replaced by the contraction or expansion of fluid elements (identified alternatively as \emph{pseudosound} or \emph{pseudo-pressure}) due to the fluctuating momentum flux, which in turn drive the acoustic field.
This conceptual simplicity makes the dilatation-based approach to the acoustic analogy particularly attractive to the experimentalist for high-speed (though subsonic), turbulent flows.
The analysis, described briefly in the subsequent section, is (relatively) simple to perform computationally, relying exclusively on second-order derivatives (unlike the third-order differential equation derived by Lilley [cite]) which include a natural filtering mechanism.
More importantly, the pseudosound field (which is the direct precursor to the source field) can be directly compared against the time-resolved hydrodynamic pressure field measured in the irrotational near-field, thus serving as a helpful validation of the computations.

\section{Ribner's Acoustic Analogy}
Starting from Lighthill's analogy (\eq{eq:lighthill_analogy}), the source term is first reduced by neglecting viscosity and entropic fluctuations (thereby assuming that the flow is of high Reynolds number and unheated), leaving only the Reynolds stress terms:
\begin{equation}
	\frac{1}{c^2}\frac{\partial^2 p}{\partial t^2} - \nabla^2 p = \nabla \cdot \nabla \cdot \rho \mathbf{u} \otimes \mathbf{u}.
\end{equation}
Ribner then split the pressure fluctuations into a propagative acoustic component and an incompressible component (pseudosound, which is associated with the convective hydrodynamic fluctuations), $p' + p_0 = p_a + p_s$. 
The incompressible component of the flow will then satisfy
\begin{equation}
	 \rho \frac{\partial}{\partial t} (\nabla \cdot \mathbf{v} )  = 0
\end{equation}
where $\mathbf{v}$ is now used in place of $\mathbf{u}$ to signify an incompressible (solenoidal) velocity field. 
Therefore, 
\begin{equation}
	- \nabla^2 p_s = \nabla \cdot \nabla \cdot \rho_0 \mathbf{v} \otimes \mathbf{v}.
\end{equation}
Ribner's analysis then assumes that the full density and velocity fields can be approximated as the incompressible (solenoidal) fields, thus producing Ribner's Dilatation Equation [ribner1962]
\begin{equation}
	\frac{1}{c^2}\frac{\partial^2 p_a}{\partial t^2} - \nabla^2 p_a = -\frac{1}{c^2}\frac{\partial^2 p_s}{\partial t^2}.
\end{equation} 

\section{Numerical Method}
\subsubsection{Helmholtz Decomposition}
	For a given vector field, $\vec{F}$, Helmholtz's theorem states that any sufficiently smooth vector field can be linearly decomposed into irrotational and solenoidal vector fields, as
	\begin{equation}
	\mathbf{F} = \mathbf{F}_{potential} + \mathbf{F}_{rotational} = \nabla \Phi + \nabla \times \Psi
	\end{equation}
	where $\Phi$ is a scalar field and $\Psi$ is a vector field.
	From basic vector calculus properties, we can therefore compute these solenoidal and irrotational components by taking the divergence of this equation, leading to:
	\begin{equation}
	\nabla \cdot \mathbf{F} = \nabla^{2} \Phi
	\end{equation}
	which is simply Poisson's equation, where in the context of velocity decomposition the forcing term is simply the divergence of the flow field.
	Since we are unable to acquire gradients in the azimuthal coordinate, we are going to drop this term, leaving
	\begin{equation}
	\frac{\partial U_r}{\partial r} + \frac{\partial U_z}{\partial z} + \frac{U_r}{r} = \frac{1}{r} \frac{\partial \Phi}{\partial r} + \frac{\partial^2 \Phi}{\partial r^2} + \frac{\partial^2 \Phi}{\partial z^2} 
	\end{equation}
	
	The solution procedure is therefore to first compute $\Phi$, then add the gradient of this field to the raw velocity vector field in order to produce the solenoidal velocity field.
\subsubsection{Pseudo-Pressure Solver}
\subsubsection{Wavelet Denoising}
\section{Wavepackets in the Pseudo-Pressure Field}