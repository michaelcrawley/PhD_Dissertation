\chapter{Dilatation as the Aeroacoustic Source}
Ribner presented an alternative approach to Lighthill's acoustic analogy which posited fluctuating fluid dilatations as the source of aeroacoustic emission [ribner 1962].
This is a reinterpretation of Lighthill's source, which in subsonic, unheated, turbulent jets consists of fluctuating momentum flux.
The driving factor behind Ribner's analysis is the conceptual simplification of the aeroacoustic sources: Lighthill's quadrupoles are replaced by the contraction or expansion of fluid elements (confusingly identified alternatively as \emph{pseudosound} or \emph{pseudo-pressure}) due to the fluctuating momentum flux, which in turn drive the acoustic field.
This conceptual simplicity makes the dilatation-based approach to the acoustic analogy particularly attractive to the experimentalist for high-speed (though subsonic), turbulent flows.
The analysis, described briefly in the subsequent section, is (relatively) simple to perform computationally, relying exclusively on second-order derivatives (unlike the third-order differential equation derived by Lilley [cite]) which include a natural filtering mechanism.
More importantly, the pseudosound field (which is the direct precursor to the source field) can be directly compared against the time-resolved hydrodynamic pressure field measured in the irrotational near-field, thus serving as a helpful validation of the computations.

\section{Ribner's Acoustic Analogy}
To briefly acquaint the reader, an overview of Ribner's analysis will be provided here.
Further details can be found in Ribner [cite] if the reader is so inclined. 
Ribner's analysis directly follows from Lighthill's, and as such includes the same restrictions on the applicable class of flows.
Starting from Lighthill's analogy (\eq{eq:lighthill_analogy}), the source term is first reduced by neglecting viscosity and entropic fluctuations (thereby assuming that the flow is of high Reynolds number and unheated), leaving only the Reynolds stress terms:
\begin{equation}
	\frac{1}{c^2}\frac{\partial^2 p}{\partial t^2} - \nabla^2 p = \nabla \cdot \nabla \cdot \rho \mathbf{u} \otimes \mathbf{u}.
\end{equation}
Ribner then split the pressure fluctuations into a propagative acoustic component and an incompressible component (pseudosound, which is associated with the convective hydrodynamic fluctuations), $p' + p_0 = p_a + p_s$. 
The incompressible component of the flow will then satisfy
\begin{equation}
	 \rho \frac{\partial}{\partial t} (\nabla \cdot \mathbf{v} )  = 0
\end{equation}
where $\mathbf{v}$ is now used in place of $\mathbf{u}$ to signify an incompressible (solenoidal) velocity field. 
Therefore, 
\begin{equation}
	- \nabla^2 p_s = \nabla \cdot \nabla \cdot \rho_0 \mathbf{v} \otimes \mathbf{v}.
	\label{eq:solenoidal_pressure}
\end{equation}
Ribner's analysis then assumes that the full density and velocity fields can be approximated as the incompressible (solenoidal) fields (the higher-order terms scaling with the \textit{fluctuating} Mach number [Ristorcelli1997]), thus producing Ribner's Dilatation Equation [ribner1962]
\begin{equation}
	\frac{1}{c^2}\frac{\partial^2 p_a}{\partial t^2} - \nabla^2 p_a = -\frac{1}{c^2}\frac{\partial^2 p_s}{\partial t^2}.
	\label{eq:ribner_source}
\end{equation} 

In this way, acoustic pressure field is ultimately linked to the time rate of change of the dilatation (see Ristorcelli [cite] for a very enlightening perturbation analysis which makes this relationship far more clear).
This is far from a controversial assertion; numerous other researchers have used the dilatation field to examine the aeroacoustic phenomena (primarily using DNS or LES simulations); see Mitchell \etal [cite 1995], Colonius \etal [cite 1997], or Freund \etal [cite 2000] for examples.
In fact, this relationship can easily be illustrated, as has been done in \fig{fig:LES_dilatation}. 
Here, phase-averaged data at two phases has been plotted from a simulated Mach 0.9 jet excited by plasma actuators at $St_{DF} = 0.25$; details of the numerical methods and results can be found in Speth \& Gaitonde [AIAA Paper 2014]. 
The toroidal structures generated by the excitation have been visualized in the foreground using Q-criterion, and the dilatation field in the background in grayscale.
Highly coherent dilatation waves can be observed convecting with the large-scale structures near the shear layer. 
Further out radially, the dilatation field becomes less coherent and more indicative of far-field propagating pressure waves.
Broadly and qualitatively speaking, the connection between the large-scale structures and the acoustic emission is clear; the current work represents a timid attempt to evaluate this relationship more formally.
\begin{figure}
	\centering
	\begin{subfigure}{.5\textwidth}
		\centering
		\includegraphics[width=0.95\linewidth]{Figures/LES_dilatation1.jpg}
		\caption{}
	\end{subfigure}%
	\begin{subfigure}{.5\textwidth}
		\centering
		\includegraphics[width=0.95\linewidth]{Figures/LES_dilatation2.jpg}
		\caption{}
	\end{subfigure}
	\caption{Two phases of actuation taken from the implicit LES database of Speth \& Gaitonde [cite]. Isosurfaces are computed from Q-criterion and colored by axial velocity, and the background corresponds to dilatation.}
	\label{fig:LES_dilatation}
\end{figure}
\section{Numerical Method}
Computing the aeroacoustic source per Ribner's dilatation method from the estimated, time-resolved velocity field constitutes a three-step process. 
First, the solenoidal velocity field is computed via the Helmholtz' decomposition; the double divergence of the resulting stress tensor is then used as the source of Poisson's equation to calculate the pseudo-pressure field. 
Finally, the pseudo-pressure field is filtered in time using an energy threshold in the wavelet domain, and the second time derivative of the resultant field is computed, producing the source field.
This process is outlined in the following sections.
\subsubsection{Helmholtz Decomposition}
For a given vector field, $\mathbf{F}$, Helmholtz's theorem states that any sufficiently smooth vector field can be linearly decomposed into irrotational and solenoidal vector fields, as
\begin{equation}
\mathbf{F} = \mathbf{F}_{potential} + \mathbf{F}_{rotational} = \nabla \Phi + \nabla \times \Psi
\end{equation}
where $\Phi$ is a scalar field and $\Psi$ is a vector field.
From basic vector calculus properties, one can therefore compute these solenoidal and irrotational components by taking the divergence of this equation, leading to:
\begin{equation}
\nabla \cdot \mathbf{F} = \nabla^{2} \Phi
\end{equation}
which is simply Poisson's equation, where in the context of a velocity decomposition the forcing term is simply the divergence of the flow field (that is, the dilatation).

This initially presents a quandary for the researcher, as only planar PIV measurements are available, and hence the azimuthal velocity and derivative terms are unknown.
As mentioned previously, the flow-field in a natural, high Reynolds number jet is a combination of numerous azimuthal Fourier modes.
Though the velocity field has been found to contain a significant amount of energy at the higher order modes, the axisymmetric mode is still the dominant mode [Glauser1987].
Additionally, it is the acoustic emission from the coherent large-scale toroidal structure generated by excitation that is the primary focus of this endeavor, not the full acoustic emission from the relatively incoherent natural turbulence.
Due to the specific nature of the excitation (axisymmetric excitation), it is not expected that the azimuthal components of the flow is significant.
Therefore, the azimuthal terms are dropped, leaving
\begin{equation}
\frac{\partial U_r}{\partial r} + \frac{\partial U_z}{\partial z} + \frac{U_r}{r} = \frac{1}{r} \frac{\partial \Phi}{\partial r} + \frac{\partial^2 \Phi}{\partial r^2} + \frac{\partial^2 \Phi}{\partial z^2}.
\label{eq:helmholtz}
\end{equation}

The solution procedure is therefore to first compute $\Phi$ using \eq{{eq:helmholtz}}, then subtract the gradient of this field (the potential velocity field) from the raw velocity vector field in order to produce the solenoidal velocity field. 
This was done using a standard, second-order accurate centered finite difference scheme by Taylor approximation.
As the flow has already been assumed to be axisymmetric, the Poisson equation is solved over the top half of the jet only.
(The top plane was chosen as the microphone tips are visible in the lower plane, which produced spurious vectors in the shear layer.)
At the lower boundary (jet centerline), a zero-normal-gradient boundary condition was used to enforce axisymmetry.
At the radial boundary, the solenoidal component was set to zero (as there is no turbulent flow in this region, and the only outgoing waves will be compressible).
At the inflow and outflow however, the proper boundary conditions are not immediately obvious.
Other researchers using numerical databases [Mansour, Unni] have argued that the solenoidal component be set to zero here as the turbulent eddies have either not grown to significant values or decay to insignificant values at these locations, respectively.
<<<<<<< HEAD
For the domain explored in the current work however, this does not appear to be necessarily accurate.
At the outflow boundary ($z/D \simeq 13$), there is still significant vortical behavior, but the Mach number is relatively low ($M_j \sim 0.5$) and hence compressibility is expected to be negligible.
=======
For the domain explored in the current work however, this does not appear to be necessarily accurate, and instead the \textit{potential} component is set to zero at both the inlet and outlet.
At the outflow boundary ($z/D \simeq 13$), there is still significant vortical behavior, but the Mach number is relatively low ($M_j \simeq 0.5$) and hence compressibility is expected to be negligible.
>>>>>>> origin/master
At the inflow boundary (which is $\sim1.4$ mm downstream of the nozzle exit), the solenoidal component is clearly not negligible in the shear layer region (since $\partial U_z / \partial r \neq 0$).
If the inflow is approximated as a plug flow (which the ensemble-averaged PIV indicates is not an unreasonable assumption), $\partial U_z / \partial z = \partial U_r / \partial r = 0$ and hence the potential component is negligible here.
Therefore, the potential component was set to zero at both the inlet and outlet.
Second-order accurate centered finite differences were again used to approximate the derivatives in the boundary conditions; a single ghost node was along each boundary to enforce these conditions.
\subsubsection{Pseudo-Pressure Solver}
Once the solenoidal velocity field is known, the fluctuating pseudo-sound field can be computed as the incompressible solution to the momentum and continuity equations (\eq{eq:solenoidal_pressure}).
As with the preceding Poisson equation, the azimuthal terms in this second Poisson equation are assumed negligible in comparison to the radial and axial terms and are thus ignored.
Again, as done previously, the governing equation is approximated using second-order accurate centered finite differences, and ghost nodes are used at the domain boundaries to enforce the boundary conditions.
Given that the hydrodynamic pressure field has been observed to strongly decay with radial position [arndt], it is assumed that the pseudo-sound fluctuations have decayed to negligible values by the time they reach the upper domain boundary.
In accordance with the boundary assumptions made in the preceding section, at the inflow boundary, plug flow has again been assumed, meaning that the pseudo-pressure \textit{fluctuations} are negligible here. 
The outflow boundary was assumed to be far enough downstream such that the fluctuations were also negligible at the boundary, and the lower boundary enforced the zero-normal-gradient assumed by axisymmetry.

\subsubsection{Wavelet Denoising}
Once the pseudo-sound field is known, the source term can be computed explicitly per \eq{eq:ribner_source}.
However, ...
\section{Wavepackets in the Pseudo-Pressure Field}