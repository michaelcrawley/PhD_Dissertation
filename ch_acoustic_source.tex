\chapter{Dilatation as the Aeroacoustic Acoustic Source}
Ribner presented an alternative approach to Lighthill's acoustic analogy which posited fluctuating fluid dilatations as the source of aeroacoustic emission.
\section{Numerical Method}
\subsubsection{Helmholtz Decomposition}
	For a given vector field, $\vec{F}$, Helmholtz's theorem states that any sufficiently smooth vector field can be linearly decomposed into irrotational and solenoidal vector fields, as
	\begin{equation}
	\vec{F} = \vec{F}_{potential} + \vec{F}_{rotational} = \nabla \Phi + \nabla \times \Psi
	\end{equation}
	where $\Phi$ is a scalar field and $\Psi$ is a vector field.
	From basic vector calculus properties, we can therefore compute these solenoidal and irrotational components by taking the divergence of this equation, leading to:
	\begin{equation}
	\nabla \cdot \vec{F} = \nabla^{2} \Phi
	\end{equation}
	which is simply Poisson's equation, where in the context of velocity decomposition the forcing term is simply the divergence of the flow field.
	Since we are unable to acquire gradients in the azimuthal coordinate, we are going to drop this term, leaving
	\begin{equation}
	\frac{\partial U_r}{\partial r} + \frac{\partial U_z}{\partial z} + \frac{U_r}{r} = \frac{1}{r} \frac{\partial \Phi}{\partial r} + \frac{\partial^2 \Phi}{\partial r^2} + \frac{\partial^2 \Phi}{\partial z^2} 
	\end{equation}
	
	The solution procedure is therefore to first compute $\Phi$, then add the gradient of this field to the raw velocity vector field in order to produce the solenoidal velocity field.
\subsubsection{Pseudo-Pressure Solver}
\subsubsection{Wavelet Denoising}
\section{Wavepackets in the Pseudo-Pressure Field}