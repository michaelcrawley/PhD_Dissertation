\chapter{The Problem To Be Solved}
\label{prob.ch}

This is the chapter that describes what cow-related problem will
be looked at in the thesis.  Basically, we want to discover the
``proper'' way to have a cow.  This will reduce the number of
inappropriate cows that occur in the United States each year.
We have studied aspects of this problem in the past, and documented it
in~\cite{bsimp00}.


That is some sort of chapter.  What a piece of work!

\section{\LaTeX\ and \BibTeX}

It is assumed that you are conversant with \LaTeX\ and \BibTeX.
Hopefully, you have made a large bibliography database as you went
through your many years at OSU, especially when you did some literature
survey as part of your General Exams. If you aren't conversant with
\BibTeX, read the relevant sections in Lamport's
book~\cite{lamport:latex}. You should learn how to do citations using
{\tt \verb+\+cite}.  Some examples of this can be found in the {\tt
.tex} files for this document (see Section~\ref{good.examples}).

\subsection{Some \LaTeX\ advice}

\LaTeX\ will also help you number your figures, etc., properly. To
reference them properly, use the {\tt \verb+\+label} and {\tt
\verb+\+ref} commands. Details can be found in \cite{lamport:latex}
and examples can be seen in the {\tt .tex} files for this document 
(see Section~\ref{good.examples}).

You should also make sure your understand the {\tt \verb+\+include} and
{\tt \verb+\+input} commands. They will allow you to break up your
documents into different parts, which you can then process separately.
That way, you don't need to print {\em everything} everytime; you can
just print a single chapter if you want ({\tt dvips} also has the
capability of printing specific pages).  However, the page numbers,
etc., will be done as though everything was printed. The sample thesis
you are looking at was split into parts and combined using the
\verb#\include# command.

\subsection{Some \BibTeX\ advice}

Everyone has opinions about how citations and references are made, so
be sure to look at lots of journals/books to decide what style you
want to use. \BibTeX\ has support for quite a few styles. The Graduate
School has been known to accept the {\tt plain} and {\tt apalike}
styles.

\subsection{Some Examples}
\label{good.examples}

Probably the {\em best} way to figure out how to use the {\tt
osudissert96} class with \LaTeXe\ is to look at how we created this
document. 

The main file ({\tt Thesis.tex}) is set up with {\em many} comments
explaining where to put any special commands you use, how to split the
document into parts and use \verb#\include# to combine the parts into
a coherent whole, what the basic order of the dissertation should be,
etc.

We also used several common ``tricks'' (\verb#\label# and \verb#\ref#,
{\tt table} and {\tt figure} environments,
\verb#\verb# and {\tt verbatim} environments, etc.) that we will make
use of in the subsequent files.  

Please read this document and look at the {\tt .tex} files to see {\em
how} we did it.  If you don't understand something, you can usually
find answers in the \LaTeX\ book \cite{lamport:latex}.  If that fails
and you still don't understand how/why something was done, ask someone
who has done it before.

\section{Other Useful Tools}

There are other tools that come in handy while writing a thesis.
Spelling Checkers and Grammar Checkers may come in handy!! On the CIS and EE
machines, {\tt ispell} is very useful. {\tt
dvips} can be used to print a subset of the pages from your
thesis. Remember not to print your entire thesis out everytime you
make a small change. Use {\tt dvips} with the -n and -p options!
