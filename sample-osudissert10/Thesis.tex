\documentclass{osudissert96}
%
% To change the dissertation to a Master's Thesis, include a documentclass
% option such as [masters], [ms], [ma], etc.  Also available are [osudraft]
% and [twoside].  As a reminder, documentclass options are a
% comma-separated list, e.g. \documentclass[ms,osudraft]{osudissert96}
%

%
% Everything between the \documentclass and the \begin{document} is
% called the preamble of the document. Everything between the
% \begin{document} and the \end{document} is called the body of the
% document. Define any additional commands you want here (the preamble)
%

% BibTeX from the BiBTeX Documentation
\def\BibTeX{{\rm B\kern-.05em{\sc i\kern-.025em b}\kern-.08em
    T\kern-.1667em\lower.7ex\hbox{E}\kern-.125emX}}

%
% It is better to break up the dissertation into multiple files (e.g.,
% one file per chapter, as well as separate files for the abstract,
% acknowledgements, and vita).  These files are brought into the
% document using \include{} statements.  There will be times, however,
% when you don't want to print the ENTIRE dissertation.  You can limit
% what will actually be printed by using the \includeonly{} statment.
% This contains a list of the files you want printed.  Any file NOT
% listed will not be printed.  However, all page numbers, references,
% etc., will be preserved as though all the files were actually
% printed. For example, the line below would result only in chapters 1
% and 3 being printed (if it were uncommented).
%

%\includeonly{ch1.intro,ch3.implem}

% UPDATED TEXT (2010):
%  In the newest format, titles should be title case everywhere.
%
% HISTORICAL TEXT (1996):
%  In the new format, the titles of each chapter should appear in
%  uppercase.  In the TOC, however, they should be in lowercase.
%  The command below automates this behavior.  However, you'll have to be
%  careful not to include \labels within your \chapter definitions or
%  there will be problems.  If you don't want this to be automated, comment
%  out the \typesetChapterTitle definition below and do your chapters in
%  the form:
%  \chapter[MY TITLE]{My Title}
%
%  \renewcommand\typesetChapterTitle[1]{\uppercase{#1}}
\renewcommand\typesetChapterTitle[1]{#1}

\begin{document}

%
% First, declare the parts of your title page 
%

\author{Bart Simpson}
\title{Have a Cow, Man (2010 Version)}
\authordegrees{C.S., D.S.}  % Degrees thus far, not including this one.
\unit{Department of Cow and Dairy Science}

\advisorname{Big Dude}
\member{Other Dude}
\member{Some other dude}
%\member{Yet another dude}      % Normally you will have advisor + 2 members


%
% The following creates the title page
%

\maketitle

% Next, EITHER a copyright or BLANK page.
%
%   The following creates a page used to copyright your dissertation
%
%   BACKGROUND: Even without this copyright page, your dissertation will
%               carry a common-law copyright. However, if your
%               dissertation ends up seeing wide distribution, your
%               common-law copyright is at risk of being expunged.
%               Adding this copyright page prevents that from happening.
%
%               There are NO DOWNSIDES to including a copyright page as
%               your document is automatically copyright by law anyway.
%               However, this copyright page is OPTIONAL. If you get rid
%               of it, uncomment the \blankpage that follows it so that
%               there is a blank page here. The graduate school requires
%               a page here that is either blank or carries the
%               copyright.
%
%   IMPORTANT NOTE: The graduate school requires either a copyright page
%                   here or a BLANK PAGE here. If you get rid of the
%                   copyright, uncomment the \blankpage that follows it.
%                   You should NOT have BOTH uncommented.
%

% If you get rid of \disscopyright, restore the \blankpage line after it
\disscopyright
%\blankpage

%
% Abstract goes here.
%

\begin{abstract}
  %  The dissertation abstract can only be 500 words.
It has been well-known within the aeroacoustic community that the dominant noise sources in high-speed turbulent jets are related to the large-scale structures which are generated in the initial shear layer by instabilities and which rapidly grow as they convect downstream.
However, the exact dynamics of these large-scale structures which are relevant to the noise generation process are less clear.
This work represents an attempt to study the dynamics and noise generated by the large-scale structures quantitatively and in high-fidelity in a Mach 0.9 turbulent jet using simultaneous pressure and velocity data acquisition systems alongside plasma-based excitation to produce coherent ring vortices in the shear layer.

In the first phase, the irrotational near-field pressure is decomposed into its constitutive acoustic and hydrodynamic components, and two-point cross-correlations are used between the acoustic near-field and far-field in order to identify the dominant noise source region.
Building upon the work of previous researchers, the decomposition is performed using a spatio-temporal wavelet transform, which was found to be more robust than previous algorithms.
Results indicated that for both individual as well as periodic large-scale structures, the dominant noise source region constitutes the upstream region of the jet, ending just before of the end of the potential core (in a time-averaged sense) in the unexcited jet.

The large-scale structure interactions were then investigated by stochastically-estimating the time-resolved velocity fields from the time-resolved near-field pressure.
For computational efficiency, the ensemble velocity snapshots were first decomposed into orthogonal modes, and the a mapping from the near-field pressure to the expansion coefficients was then produced using a feedforward neural network using backpropagation for learning.
The coherent structures generated by the excitation were then identified and tracked using standard vortex identification routines.
For the impulsively-excited jet, the individual structures quickly rolled up into a coherent structure within two jet diameters and then advected until roughly four jet diameters downstream, at which point it underwent a rapid disintegration.
For the periodically-excited jet, multiple smaller-scale structures are initially produced; these quickly merge into a single large-scale structure which matches the excitation wavelength.
Similar to the impulsively-excited structures, these now large-scale structures advect downstream and undergo a rapid disintegration near the end of the potential core. 

Finally, from Ribner's dilatation-based acoustic analogy the aeroacoustic source terms were computed using the time-resolved velocity field produced by the stochastic estimation.
Interpretation of the results is limited however, due to the number of assumptions and simplifications necessary for the computations, given the realities of the available experimental facilities.
Analysis of the computed source fields identified the coherent structures producing a convected wavepacket-like event, centered on the jet lipline though reaching into the potential core.
For the individual vortex rings, a clear modulation of the spatial extent and amplitude was observed as the vortex began to break down just upstream of the end of the potential core.
This behavior is also present for the periodic train of vortices, however it is obscured by an amplification of the source in the upstream region, corresponding to the pairing location for the multiple smaller-scale structures generated by the excitation. 



\end{abstract}


%
% UPDATED TEXT (2010):
%  The graduate school does not require an external abstract. If this
%  changes, follow the old instructions below.
%
% HISTORICAL TEXT (1996):
%  Uncomment the three lines below to generate the external abstract.  Two
%  copies of this must be turned in to the graduate school.  These lines can
%  be placed pretty much anywhere, since the page numbering should be
%  independent of the rest of the thesis
%

% \begin{externalabstract}
%   %  The dissertation abstract can only be 500 words.
It has been well-known within the aeroacoustic community that the dominant noise sources in high-speed turbulent jets are related to the large-scale structures which are generated in the initial shear layer by instabilities and which rapidly grow as they convect downstream.
However, the exact dynamics of these large-scale structures which are relevant to the noise generation process are less clear.
This work represents an attempt to study the dynamics and noise generated by the large-scale structures quantitatively and in high-fidelity in a Mach 0.9 turbulent jet using simultaneous pressure and velocity data acquisition systems alongside plasma-based excitation to produce coherent ring vortices in the shear layer.

In the first phase, the irrotational near-field pressure is decomposed into its constitutive acoustic and hydrodynamic components, and two-point cross-correlations are used between the acoustic near-field and far-field in order to identify the dominant noise source region.
Building upon the work of previous researchers, the decomposition is performed using a spatio-temporal wavelet transform, which was found to be more robust than previous algorithms.
Results indicated that for both individual as well as periodic large-scale structures, the dominant noise source region constitutes the upstream region of the jet, ending just before of the end of the potential core (in a time-averaged sense) in the unexcited jet.

The large-scale structure interactions were then investigated by stochastically-estimating the time-resolved velocity fields from the time-resolved near-field pressure.
For computational efficiency, the ensemble velocity snapshots were first decomposed into orthogonal modes, and the a mapping from the near-field pressure to the expansion coefficients was then produced using a feedforward neural network using backpropagation for learning.
The coherent structures generated by the excitation were then identified and tracked using standard vortex identification routines.
For the impulsively-excited jet, the individual structures quickly rolled up into a coherent structure within two jet diameters and then advected until roughly four jet diameters downstream, at which point it underwent a rapid disintegration.
For the periodically-excited jet, multiple smaller-scale structures are initially produced; these quickly merge into a single large-scale structure which matches the excitation wavelength.
Similar to the impulsively-excited structures, these now large-scale structures advect downstream and undergo a rapid disintegration near the end of the potential core. 

Finally, from Ribner's dilatation-based acoustic analogy the aeroacoustic source terms were computed using the time-resolved velocity field produced by the stochastic estimation.
Interpretation of the results is limited however, due to the number of assumptions and simplifications necessary for the computations, given the realities of the available experimental facilities.
Analysis of the computed source fields identified the coherent structures producing a convected wavepacket-like event, centered on the jet lipline though reaching into the potential core.
For the individual vortex rings, a clear modulation of the spatial extent and amplitude was observed as the vortex began to break down just upstream of the end of the potential core.
This behavior is also present for the periodic train of vortices, however it is obscured by an amplification of the source in the upstream region, corresponding to the pairing location for the multiple smaller-scale structures generated by the excitation. 



% \end{externalabstract}

%
%  My Dedication
%

\dedication{This is dedicated to the one I love \ldots\ la la la \ldots}


%
% Bring in Acknowledgement and Vita from separate files named ``ack.tex''
% and ``vita.tex''.
%

\begin{acknowledgements}
I should probably acknowledge someone here \ldots
\end{acknowledgements}


\begin{vita}

\dateitem{September 10, 1986}{Born - Plano, Texas}

\dateitem{2009}{B.S. Mechanical Engineering, \\
				University of Texas, Austin.}

\dateitem{2009-present}{Graduate Research Associate,\\
			 The Ohio State University.}


\begin{publist}

%% UPDATE FOR 2010:
%  Grad school only wants research publications, and it only wants those
%  research pubs that are actually published. Accepted or ``to appear''
%  publications don't count. If they look closely, they'll tell you to
%  remove any publications that aren't in print. Haivng said that, they
%  probably won't look that closely unless you put a really long list
%  here. You're tempting fate if you add instructional publications
%  though.

\researchpubs

\pubitem{\textbf{M. Crawley}, C.-W. Kuo, and M. Samimy,
\newblock ``Identification of the Acoustic Response in the Irrotational Near-field of an Excited Subsonic Jet.''
\newblock submitted to \emph{International Journal of Aeroacoustics}.}
	

\pubitem{\textbf{M. Crawley}, R. Speth, D. V. Gaitonde, and M. Samimy, 
\newblock ``A Study of the Noise Source Mechanisms in an Excited Mach 0.9 Jet - Complementary Experimental and Computational Analysis.''
\newblock AIAA Paper 2015-0736, \emph{53$^{rd}$ AIAA Aerospace Sciences Meeting}.}
	

\pubitem{\textbf{M. Crawley}, A. Sinha, and M. Samimy, 
\newblock ``Near-field and Acoustic Far-field Response of a High-Speed Jet Forced with Plasma Actuators.''
\newblock \emph{AIAA Journal}, expected 2015.}
	

\pubitem{\textbf{M. Crawley} and M. Samimy, 
\newblock ``Decomposition of the Near-Field Pressure in an Excited Subsonic Jet.''
\newblock AIAA Paper 2014-2342, \emph{20$^{th}$ AIAA/CEAS Aeroacoustics Conference}.}
	

\pubitem{\textbf{M. Crawley}, A. Sinha, and M. Samimy, 
\newblock ``Near-field Pressure and Far-field Acoustic Response of Forced High-Speed Jets.''
\newblock AIAA Paper 2014-0527, \emph{52$^{nd}$ AIAA Aerospace Sciences Meeting}.}	
	

\pubitem{\textbf{M. Crawley}, H. Alkandry, A. Sinha, and M. Samimy, 
\newblock ``Correlation of Irrotational Near-Field Pressure and Far-Field Acoustic in Forced High-Speed Jets.''
\newblock AIAA Paper 2013-2188, \emph{19$^{th}$ AIAA/CEAS Aeroacoustics Conference}.}
	

\pubitem{H. Alkandry, \textbf{M. Crawley}, A. Sinha, M. Kearney-Fischer, and M. Samimy,
\newblock ``An Investigation of the Irrotational Near Field of an Excited High-Speed Jet.''
\newblock AIAA Paper 2013-0325, \emph{51$^{st}$ AIAA Aerospace Sciences Meeting}.}
	

\pubitem{\textbf{M. Crawley}, M. Kearney-Fischer, and M. Samimy, 
\newblock ``Control of a Supersonic Rectangular Jet Using Plasma Actuators.''
\newblock AIAA Paper 2012-2211, \emph{18$^{th}$ AIAA/CEAS Aeroacoustics Conference}.}
\end{publist}



\begin{fieldsstudy}

% The \majorfield* uses the unit specified in the \unit command used
% earlier in your document. If you want to use a different unit, use the
% second form shown here
%\majorfield*
\majorfield{Mechanical and Aerospace Engineering}

%%
%% Note:  If there were only one field of study, the following list 
%%        would best be done using the following command:
%%
%%  \onestudy{Only Topic}{Only Professor}
%%

% \begin{studieslist}
% \studyitem{Topic 1}{Prof.\ Big Dude}
% \studyitem{Topic 2}{Prof.\ Other Dude}
% \studyitem{Topic 3}{Prof.\ Another Dude}
% \end{studieslist}

\end{fieldsstudy}

\end{vita}




%
% Make the Table of Contents and other good stuff
%

\tableofcontents
\listoftables
\listoffigures


%
% The following is a list of chapters.  Each is brought in from a
% separate file using the \include{} command.
%

\chapter{Introduction}
\label{intro.ch}

This is the first chapter of the dissertation. It probably rambles on
about Cows and Bulls and what-not.  It most likely contains no
details, but suggests what will be seen in the other chapters.  Maybe
you can give a glimpse of ``the problem'' in this chapter, state your
thesis, and suggest how your thesis is going to be justified. Since
the bulk of the material is going to be in the succeeding chapters,
this chapter should inform the reader about what the subsequent
chapters contain.

Pretty cool, huh?

\section{Getting Started}

Well, before you write up your dissertation and zoom over to the
Graduate School with your {\em magnus opus}, be sure to read the Grad
School's Graduate Student Handbook~\cite{osu:guidelines}. It tells you a lot of
stuff you need to know. And since you will be using \LaTeX\ to write
your dissertation, have Lamport's \LaTeXe\ book~\cite{lamport:latex}
handy.

\section{The OSU Dissertation Class}

The latest version of the OSU Dissertation Class (1996) was derived from
the old version originally created in the CIS department.  The new format
specifications are much closer to the way the classes in \LaTeXe\ function
by default; therefore, many of the more challenging aspects of creating the
class are no longer needed.

A lot of graduate students have contributed to the
creation of the dissertation class files {\tt osudissert96.cls} and {\tt
osudissert96-mods.sty}. Elizabeth Zwicky created the original style files. J.
Ramanujam and Con O'Connell are the other main contributors.  Most
recently, Mark Hanes has updated the files to conform to the 1996
Graduate School guidelines.  Al Fencl created
the vita style in {\tt osudissert96}. All other contributors are mentioned
at the top of that file.

This document was initially created by Manas Mandal from Con O'Connell's
dissertation to help a student in Physics.  Since a lot of other people
have expressed interest in the style files and how to use them in a
dissertation, this sample dissertation was produced by Manas Mandal and Al
Fencl.  This document has also been modified by Mark Hanes to reflect the
1996 Graduate School requirements.

\subsection{What you need in order to use {\tt osudissert96}}

To use the {\tt osudissert96} class, you will need to have
%
\begin{enumerate}
\item \LaTeXe\ (obviously).
\item {\tt osudissert96.cls}
\item {\tt osudissert96-mods.sty}
\end{enumerate}
These files are currently all available on the EE HP and Sun Workstations.
%

\subsection{Compiling the Example}
\label{compile.example}

This document is split into several files.  If you have not compiled
it yet or had difficulty compiling it, you should make sure you have
the following files:
%
\begin{center}
\begin{tabular}{l l}
{\tt Thesis.tex} & The main document. Run \LaTeX\ on this file\\
{\tt abstract.tex} & The Abstract for the thesis.\\
{\tt ack.tex} & The Acknowledgement for the thesis.\\
{\tt vita.tex} & The Vita for the author of the thesis.\\
{\tt ch1.intro.tex} & Chapter 1 of the thesis.\\
{\tt ch2.problem.tex} & Chapter 2 of the thesis.\\
{\tt ch3.implem.tex} & Chapter 3 of the thesis.\\
{\tt ch4.end.tex} & Chapter 4 of the thesis.\\
{\tt app1.tex} & The first appendix of the thesis.\\
{\tt app2.tex} & The second appendix of the thesis.\\
{\tt bibfile.bib} & The sample bibliography database.
\end{tabular}
\end{center}
%

To fully compile this example, you should do the following:
%
\begin{enumerate}
\item Run \LaTeX\ on {\tt Thesis}.  This will do the inital
compilation of the document and will create a list of the labels and
references made.
%
\item Run \BibTeX\ on {\tt Thesis}.  This will go into {\tt
bibfile.bib} and extract the appropriate bibliography for the
references  cited in the dissertation.
%
\item Run \LaTeX\ on {\tt Thesis} {\em two} more times.  
The first time, \LaTeX\ will go through and (at the end) will
recognize the references made in the citations and will set up the
table of contents. However, the table of contents will probably be off
since the table of contents will grow.  The second time through,
\LaTeX\ will get the page numbers correct in the table of contents.
\end{enumerate}
%
You will need to perform the above steps on your own
dissertation/thesis as well.

\subsection{Getting more information}

If you need additional information, please check out EE's \LaTeXe\ web
page, {\verb+http://eewww.eng.ohio-state.edu/~hanes/latex2e+}.  If you
can't find what you need there, you might want to read the {\tt .cls} and
{\tt .sty} files used
to generate this dissertation to see how the various commands are used and
start from there. A complete list of the commands defined in {\tt
osudissert96} is also provided in Appendix~\ref{allcommands:app}.

\subsection{Ph.D. Dissertation and Master's Thesis}

The {\tt osudissert96} class provides support for both Ph.D.
dissertations and Master's theses. While this document is an example
of a Ph.D.  dissertation, it is possible generate a Master's thesis
just by including the appropriate documentclass option.  For example, to
produce a Master's of Science thesis, give the option {\tt ms}:

\begin{center}
{\verb+\documentclass[ms]{osudissert96}+}
\end{center}

\section{Organization of this Thesis}

The rest of this thesis is organized as follows. 

Chapter~\ref{prob.ch} will introduce the problems with cows, and what
all has been done by other researchers about it.  In reality,
Chapter~\ref{prob.ch} discusses \LaTeX\ and provides pointers to
advice and examples of how to use the {\tt osudissert96} class.

Chapter~\ref{implem.ch} describes the details of the implementation
method used in having a cow, and how it does solve all the world's
problems. In reality, Chapter~\ref{implem.ch} discusses figures and
tables and how to create them ``easily'' using \LaTeX.

Chapter~\ref{end.ch} summarizes the results of the thesis, and gives
pointers to future research that can be based on this exemplary work.
It has no real bearing on reality.

Appendix~\ref{data.app} explains some of the data used to create
Table~\ref{example-table}. It also has little to do with reality.

Appendix~\ref{allcommands:app} lists all the commands defined in
{\tt osudissert96}.
     % tell them what you are going to tell them 
\chapter{The Problem To Be Solved}
\label{prob.ch}

This is the chapter that describes what cow-related problem will
be looked at in the thesis.  Basically, we want to discover the
``proper'' way to have a cow.  This will reduce the number of
inappropriate cows that occur in the United States each year.
We have studied aspects of this problem in the past, and documented it
in~\cite{bsimp00}.


That is some sort of chapter.  What a piece of work!

\section{\LaTeX\ and \BibTeX}

It is assumed that you are conversant with \LaTeX\ and \BibTeX.
Hopefully, you have made a large bibliography database as you went
through your many years at OSU, especially when you did some literature
survey as part of your General Exams. If you aren't conversant with
\BibTeX, read the relevant sections in Lamport's
book~\cite{lamport:latex}. You should learn how to do citations using
{\tt \verb+\+cite}.  Some examples of this can be found in the {\tt
.tex} files for this document (see Section~\ref{good.examples}).

\subsection{Some \LaTeX\ advice}

\LaTeX\ will also help you number your figures, etc., properly. To
reference them properly, use the {\tt \verb+\+label} and {\tt
\verb+\+ref} commands. Details can be found in \cite{lamport:latex}
and examples can be seen in the {\tt .tex} files for this document 
(see Section~\ref{good.examples}).

You should also make sure your understand the {\tt \verb+\+include} and
{\tt \verb+\+input} commands. They will allow you to break up your
documents into different parts, which you can then process separately.
That way, you don't need to print {\em everything} everytime; you can
just print a single chapter if you want ({\tt dvips} also has the
capability of printing specific pages).  However, the page numbers,
etc., will be done as though everything was printed. The sample thesis
you are looking at was split into parts and combined using the
\verb#\include# command.

\subsection{Some \BibTeX\ advice}

Everyone has opinions about how citations and references are made, so
be sure to look at lots of journals/books to decide what style you
want to use. \BibTeX\ has support for quite a few styles. The Graduate
School has been known to accept the {\tt plain} and {\tt apalike}
styles.

\subsection{Some Examples}
\label{good.examples}

Probably the {\em best} way to figure out how to use the {\tt
osudissert96} class with \LaTeXe\ is to look at how we created this
document. 

The main file ({\tt Thesis.tex}) is set up with {\em many} comments
explaining where to put any special commands you use, how to split the
document into parts and use \verb#\include# to combine the parts into
a coherent whole, what the basic order of the dissertation should be,
etc.

We also used several common ``tricks'' (\verb#\label# and \verb#\ref#,
{\tt table} and {\tt figure} environments,
\verb#\verb# and {\tt verbatim} environments, etc.) that we will make
use of in the subsequent files.  

Please read this document and look at the {\tt .tex} files to see {\em
how} we did it.  If you don't understand something, you can usually
find answers in the \LaTeX\ book \cite{lamport:latex}.  If that fails
and you still don't understand how/why something was done, ask someone
who has done it before.

\section{Other Useful Tools}

There are other tools that come in handy while writing a thesis.
Spelling Checkers and Grammar Checkers may come in handy!! On the CIS and EE
machines, {\tt ispell} is very useful. {\tt
dvips} can be used to print a subset of the pages from your
thesis. Remember not to print your entire thesis out everytime you
make a small change. Use {\tt dvips} with the -n and -p options!
   % describe the problem statement
\chapter{How to Actually Have a Cow}
\label{implem.ch}

To actually have a cow, one must read~\cite{simpson:cows} and look at
the Cow Car in Figure~\ref{example-figure}. The table shown in
Table~\ref{example-table} was generated using the data in
Appendix~\ref{data.app} and further justifies this research.

%%
%% A sample figure
%%

\begin{figure}
\setlength{\unitlength}{1mm}
\begin{picture}(30,30)(-50,0)
\put(15,20){\circle{6}}
\put(30,20){\circle{6}}
\put(15,20){\circle*{2}}
\put(30,20){\circle*{2}}
\put(10,24){\framebox(25,8){Cow Car}}
\put(10,32){\vector(-2,1){10}}
\end{picture}
\caption{\label{example-figure}An example figure.}
\end{figure}

%%
%% A sample table
%%

\begin{table}
\begin{center}
\begin{tabular}{|| r | c ||}
\hline\hline
\multicolumn{2}{|| c ||}{Summary of Cows Had}\\
\hline
\multicolumn{1}{|| c |}{Year} & \multicolumn{1}{c ||}{Number of Cows}\\
\hline
1990 & 57\\
\hline
1991 & 80\\
\hline
1992 & 199\\
\hline\hline
\end{tabular}
\end{center}
\caption{\label{example-table}An example table.}
\end{table}


%%
%% Another sample table
%%

\begin{table}
\begin{center}
\begin{tabular}{|| r | c ||}
\hline\hline
\multicolumn{2}{|| c ||}{Summary of Cows Had}\\
\hline
\multicolumn{1}{|| c |}{Year} & \multicolumn{1}{c ||}{Number of Cows}\\
\hline
1990 & 57\\
\hline
1991 & 80\\
\hline
1992 & 199\\
\hline\hline
\end{tabular}
\end{center}
\caption{\label{another-example-table}Same example table but with an
unnecessarily long name so as to cause it to go to multiple lines in the
List of Figures.} 
\end{table}



\section{Captions of Figures and Tables in \LaTeX}
\label{latex-captions}

In both figures and tables, you will want to have some sort of caption
describing your work.  These captions are generated using the command
\verb#\caption{#{\em text}\verb#}# where {\em text} is your
table/figure description.  When this command is placed in a figure
(see Section~\ref{latex-figures}), it will generate the output
``Figure {\em M.N}: {\em text}'' where {\em M} is the chapter number and
{\em N} is the sequence number of
this figure (starting with Figure~1.1) and {\em text} is the text you
have specified.  If the \verb#\caption# command is used in a table
environment (see Section~\ref{latex-tables}), you will get ``Table
{\em M.N}: {\em text}'' where {\em M} is the chapter number and {\em N} is
the sequence number of this table (starting with Table~1.) and {\em text} is
the text you have specified.

One important thesis style note (see~\cite{osu:guidelines}) is that
the captions in figures {\em and} tables should come {\em after} the
figure/table.  This is different than in previous format specifications
from the Graduate School.


\section{Figures in \LaTeX}
\label{latex-figures}
Figures are included in \LaTeX\ using the {\tt figure} environment:
%
\begin{center}
\begin{tabular}{l}
\verb#\begin{figure}#\\
{\em stuff for your figure}\\
\verb#\caption{#Description of your figure\verb#}#\\
\verb#\end{figure}#
\end{tabular}
\end{center}
%
Normally it is a pain to draw figures using \TeX/\LaTeX\ commands. The
picture shown in Figure~\ref{example-figure} was produced using the
{\tt picture} environment.

\subsection{``Pasting'' Existing  Figures in \LaTeX}
\label{pasted-figures}

Occasionally, you will want/need to include figures and pictures
generated by other people.  To do this, you will basically want to
have \LaTeX\ leave some blank space for you and then (after the
document is printed), use a copy machine to ``paste'' your figure into
the document.  This is easily done by using the \verb#\vspace*#
command.  This command will leave a specified amount of vertical space
in your document.  For example, to leave 3 inches of vertical space,
you would type
%
\begin{center}
\begin{tabular}{l}
\verb#\vspace*{3in}#
\end{tabular}
\end{center}
%

If you do make use of someone else's figures, you {\em must} have
written permission from the creator of the figure to use it in your
dissertation.  {\em Be sure you have all required permissions before
you attempt to turn in your thesis to the graduate school!!}
Permission via email is usually adequate.

\subsection{Postscript figures in \LaTeX}
\label{ps-figures}

\begin{quotation}
Note: This section is written assuming you are using a {\em
PostScript} printer such as those available in CIS and EE. If you are using
\LaTeXe\ in another department, please check with the system staff there to make
sure this section applies to you.
\end{quotation}

A lot of people prefer to generate their pictures using
WYSIWYG\footnote{{\em W}hat {\em Y}ou {\em S}ee {\em I}s {\em W}hat {\em
Y}ou {\em G}et.} programs like {\em idraw}, {\em xfig}, {\em MacDraw}, {\em
FrameMaker}, etc. These result in a {\em PostScript} file which contains
the picture. Figures in postscript can also be produced as the output of a
graphics program, or as the dump of a window in a windowing system. These
can be included in your document using the \LaTeXe\ package {\tt epsfig}.
In the portion where the ``{\em stuff for your figure}'' goes, use the {\tt
\verb+\+epsfig} command to include the postscript file.  When the document
is finally printed on a postscript printer using the {\tt dvips} program,
the postscript file will be printed out as part of the document.

\subsubsection{The Recommended Method}
\label{epsf-method}

The recommended method is as follows. It uses the package
{\tt epsfig}, and assumes that the postscript file conforms to the
{\em Encapsulated Postscript Script} standards. This calculates the
amount of space the figure will take up, and automatically reserves
space within the document for the postscript figure.  The command
{\tt \verb+\+epsfig} can be used as follows:\\
\hspace*{3em}{\tt \verb+\epsfig{file=psfilename,width=desiredwidth}+}\\
An example command would be\\
\hspace*{3em}{\tt \verb+\epsfig{file=dsmfig.ps,width=4in}+}\\
where {\tt dsmfig.ps} is the name of the postscript file.


\newpage
\section[Tables in \LaTeX]{Tables in \LaTeX}
\label{latex-tables}

Tables are created using the {\tt table} environment:
%
\begin{center}
\begin{tabular}{l}
\verb#\begin{table}#\\
{\em stuff for your table (using tabular?)}\\
\verb#\caption{#Description of your table\verb#}#\\
\verb#\end{table}#
\end{tabular}
\end{center}
%
Most tables are created using the {\tt tabular} environment. Details
on using the {\tt tabular} environment can be found in Lamport's
\LaTeXe\ book~\cite{lamport:latex}.  An example table is shown in
Table~\ref{example-table}.

Of course, you can also use the method of Section~\ref{pasted-figures}
to leave blank space and ``paste'' in a pre-made table after the
document is printed as well.

    % what is your solution
\chapter{Contributions and Future Work}
\label{end.ch}

We have successfully shown in Chapter~\ref{implem.ch} how the problems
of having a cow can be solved. A lot of work can still be done in this
field so that better and bigger cows can be had. This is a matter
which can be pursued by those weeny grad students who have just
started their Ph.D. 

In reality, this document has been prepared by Manas Mandal and Al
Fencl to assist people in using the {\tt osudissert96} class for
\LaTeX\ to generate masters theses and doctoral dissertations at the Ohio
State University. The relevant files are available from the Department
of Computer and Information Science or Electrical Engineering.  Also, ask
around your department; the files may have already migrated to a computer
in your home department.

Keep in mind that this document {\em is not intended to teach you
\LaTeX\ } nor should it be used as such.  Lamport's book
\cite{lamport:latex} is a very good introduction to the wonderful
world of \LaTeX\ and it would be well worth your while to purchase
that book!

It is our sincere hope that the use of this document and the related
style files will help you in producing your thesis and/or
dissertation. We wish you ``Good Luck''!
       % conclusion stuff


%
% If you have appendices in your dissertation, you will need the
% following, else keep it commented. The following appendices are in
% files called ``app1.tex'', and ``app2.tex'', and they
% look just like any chapter.
%

\appendix
\chapter{The Data on Cows}
\label{data.app}

This is the data that was used to produce the table in
Table~\ref{example-table}. In 1990, 57 people had cows. In 1991, 80
people had cows. In 1992, 199 people had cows.

\chapter{Important commands defined in {\tt osudissert96}}
\label{allcommands:app}

The following is a list of all commands available in {\tt osudissert96}.


{\baselineskip=15pt
\begin{verbatim}
\author{First Middle Last}
\title{The title of the thesis}
\authordegrees{degree1, degree2}
\unit{Department of Whatever The Name Is}
\degree{Doctor of Philosophy}
\committee{Dissertation}
\advisorname{Name of advisor}  % Possible usage "Prof. Big Dude"
\member{Name of committee member}

\thesis  % this makes it a thesis rather than a dissertation,
         %  similar to the documentclass option [masters] or [ms].

\maketitle

\disscopyright      % or \blankpage

\begin{abstract}
\end{abstract}

\begin{externalabstract}
\end{externalabstract}


\dedication{This is dedicated to \ldots}

\begin{acknowledgements}
\end{acknowledgements}

\begin{vita}
\dateitem{Important Date}{ Why its important}

\begin{publist}
\researchpubs
\pubitem{Bibliography item (from BibTeX?)}

\instructpubs
\pubitem{Bibliography item (from BibTeX?)}
\end{publist}

\begin{fieldsstudy}
\majorfield*        % which uses \unit above
% \majorfield{Your Major Field}
\begin{studieslist}
\studyitem{Topic 1}{Prof.\ 1}
\studyitem{Topic 2}{Prof.\ 2}
\studyitem{Topic 3}{Prof.\ 3}
\end{studieslist}
%% Note:  If there were only one field of study, the following list 
%%        would best be done using the following command:
%%  \onestudy{Only Topic}{Only Professor}
\end{fieldsstudy}

\end{vita}
\end{verbatim}
}



%
% The all important bibliography file at the end of your document!! Use
% the bibstyle you (your department) like in the \bibliographystyle{}
% statement and list the name of your bibliography database file in
% the \bibliography{} statement.  In this example, ``bibfile.bib'' is
% the name of the database.  See the LaTeX manual appendix B for details
% about the bibliography database and BibTeX.
%

\bibliographystyle{plain}
\bibliography{bibfile}

\end{document}




