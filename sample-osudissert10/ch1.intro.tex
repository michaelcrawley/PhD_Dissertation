\chapter{Introduction}
\label{intro.ch}

This is the first chapter of the dissertation. It probably rambles on
about Cows and Bulls and what-not.  It most likely contains no
details, but suggests what will be seen in the other chapters.  Maybe
you can give a glimpse of ``the problem'' in this chapter, state your
thesis, and suggest how your thesis is going to be justified. Since
the bulk of the material is going to be in the succeeding chapters,
this chapter should inform the reader about what the subsequent
chapters contain.

Pretty cool, huh?

\section{Getting Started}

Well, before you write up your dissertation and zoom over to the
Graduate School with your {\em magnus opus}, be sure to read the Grad
School's Graduate Student Handbook~\cite{osu:guidelines}. It tells you a lot of
stuff you need to know. And since you will be using \LaTeX\ to write
your dissertation, have Lamport's \LaTeXe\ book~\cite{lamport:latex}
handy.

\section{The OSU Dissertation Class}

The latest version of the OSU Dissertation Class (1996) was derived from
the old version originally created in the CIS department.  The new format
specifications are much closer to the way the classes in \LaTeXe\ function
by default; therefore, many of the more challenging aspects of creating the
class are no longer needed.

A lot of graduate students have contributed to the
creation of the dissertation class files {\tt osudissert96.cls} and {\tt
osudissert96-mods.sty}. Elizabeth Zwicky created the original style files. J.
Ramanujam and Con O'Connell are the other main contributors.  Most
recently, Mark Hanes has updated the files to conform to the 1996
Graduate School guidelines.  Al Fencl created
the vita style in {\tt osudissert96}. All other contributors are mentioned
at the top of that file.

This document was initially created by Manas Mandal from Con O'Connell's
dissertation to help a student in Physics.  Since a lot of other people
have expressed interest in the style files and how to use them in a
dissertation, this sample dissertation was produced by Manas Mandal and Al
Fencl.  This document has also been modified by Mark Hanes to reflect the
1996 Graduate School requirements.

\subsection{What you need in order to use {\tt osudissert96}}

To use the {\tt osudissert96} class, you will need to have
%
\begin{enumerate}
\item \LaTeXe\ (obviously).
\item {\tt osudissert96.cls}
\item {\tt osudissert96-mods.sty}
\end{enumerate}
These files are currently all available on the EE HP and Sun Workstations.
%

\subsection{Compiling the Example}
\label{compile.example}

This document is split into several files.  If you have not compiled
it yet or had difficulty compiling it, you should make sure you have
the following files:
%
\begin{center}
\begin{tabular}{l l}
{\tt Thesis.tex} & The main document. Run \LaTeX\ on this file\\
{\tt abstract.tex} & The Abstract for the thesis.\\
{\tt ack.tex} & The Acknowledgement for the thesis.\\
{\tt vita.tex} & The Vita for the author of the thesis.\\
{\tt ch1.intro.tex} & Chapter 1 of the thesis.\\
{\tt ch2.problem.tex} & Chapter 2 of the thesis.\\
{\tt ch3.implem.tex} & Chapter 3 of the thesis.\\
{\tt ch4.end.tex} & Chapter 4 of the thesis.\\
{\tt app1.tex} & The first appendix of the thesis.\\
{\tt app2.tex} & The second appendix of the thesis.\\
{\tt bibfile.bib} & The sample bibliography database.
\end{tabular}
\end{center}
%

To fully compile this example, you should do the following:
%
\begin{enumerate}
\item Run \LaTeX\ on {\tt Thesis}.  This will do the inital
compilation of the document and will create a list of the labels and
references made.
%
\item Run \BibTeX\ on {\tt Thesis}.  This will go into {\tt
bibfile.bib} and extract the appropriate bibliography for the
references  cited in the dissertation.
%
\item Run \LaTeX\ on {\tt Thesis} {\em two} more times.  
The first time, \LaTeX\ will go through and (at the end) will
recognize the references made in the citations and will set up the
table of contents. However, the table of contents will probably be off
since the table of contents will grow.  The second time through,
\LaTeX\ will get the page numbers correct in the table of contents.
\end{enumerate}
%
You will need to perform the above steps on your own
dissertation/thesis as well.

\subsection{Getting more information}

If you need additional information, please check out EE's \LaTeXe\ web
page, {\verb+http://eewww.eng.ohio-state.edu/~hanes/latex2e+}.  If you
can't find what you need there, you might want to read the {\tt .cls} and
{\tt .sty} files used
to generate this dissertation to see how the various commands are used and
start from there. A complete list of the commands defined in {\tt
osudissert96} is also provided in Appendix~\ref{allcommands:app}.

\subsection{Ph.D. Dissertation and Master's Thesis}

The {\tt osudissert96} class provides support for both Ph.D.
dissertations and Master's theses. While this document is an example
of a Ph.D.  dissertation, it is possible generate a Master's thesis
just by including the appropriate documentclass option.  For example, to
produce a Master's of Science thesis, give the option {\tt ms}:

\begin{center}
{\verb+\documentclass[ms]{osudissert96}+}
\end{center}

\section{Organization of this Thesis}

The rest of this thesis is organized as follows. 

Chapter~\ref{prob.ch} will introduce the problems with cows, and what
all has been done by other researchers about it.  In reality,
Chapter~\ref{prob.ch} discusses \LaTeX\ and provides pointers to
advice and examples of how to use the {\tt osudissert96} class.

Chapter~\ref{implem.ch} describes the details of the implementation
method used in having a cow, and how it does solve all the world's
problems. In reality, Chapter~\ref{implem.ch} discusses figures and
tables and how to create them ``easily'' using \LaTeX.

Chapter~\ref{end.ch} summarizes the results of the thesis, and gives
pointers to future research that can be based on this exemplary work.
It has no real bearing on reality.

Appendix~\ref{data.app} explains some of the data used to create
Table~\ref{example-table}. It also has little to do with reality.

Appendix~\ref{allcommands:app} lists all the commands defined in
{\tt osudissert96}.
