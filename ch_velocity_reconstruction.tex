\chapter{Estimation of Time-resolved Velocity Fields}
Analysis of the evolution and interaction of the large-scale structures, and ultimately the noise generated thereby, is greatly simplified by the acquisition of time-resolved flow-field measurements.
(In fact, as will be discussed in \sect{sect:source}, computation of the aeroacoustic source field will eventually require a temporal derivative and thus time-resolved data is a necessity for the current work.)
Unfortunately, directly acquiring time-resolved velocity fields for the jet currently under study is simply not possible due to the combination of a large domain of interest ($0 \leq z/D \lesssim 12, 0 \leq r/D \lesssim 3$) and high characteristic frequencies on the order of several kilohertz.
Full-field, high-fidelity measurement techniques capable of this repetition rate simply do not exist at present.

Phase-locking of a data acquisition system to actuators (or a naturally occurring resonance tone) is a common experimental technique [Kearney-Fischer2011]; by varying the delay between the trigger and time of data acquisition, multiple phases can be acquired and the coherent component of the phenomena can be analyzed.
Phase-locking of the PIV system to the LAFPAs was initially considered for the present work, but quickly discarded.
Sample analysis performed using a numerical database indicated that a very high temporal resolution was required in order to accurately compute fluctuations in the dilatation field (the relevance of which will become more apparent in the following chapter).
At moderate to high excitation frequencies, this was feasible, though potentially tedious (for example, $\sim$16 phases were estimated as necessary at $St_{DF} =0.25$).
\section{Stochastic Estimation}
\subsection{Artificial Neural Networks}
\subsection{Comparison Against Phase-locking}
\section{Large-Scale Structure Interaction}