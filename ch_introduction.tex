\chapter{Introduction}
\label{introduction}

The advent of the turbojet engine led to a transformation in both commercial and military aviation, allowing for much faster flight than previously possible with propellor-driven aircraft. 
However, the increased thrust of turbojets has come at great cost.
Significant acoustic radiation is generated by the rotating components (compressor, turbine, fan), by the combustion process, and ultimately by the free jet itself. 
On the commercial side, the escalating number of flights, encroachment of urban and residential areas near airports, and tightening of environmental regulations have combined to force airports to institute curfews, surcharges and flight path restrictions to combat noise pollution. 
For the military, hearing damage inflicted on nearby personnel (particularly on aircraft carriers) has necessitated the implementation of noise reduction concepts on tactical aircraft.
During takeoff and landing, when acoustic radiation is most problematic to ground crew and  surrounding urban and residential areas, the dominant noise source of the jet engine is the aeroacoustic radiation generated by the high velocity engine exhaust.
This has spurred extensive research, spanning over six decades, into the aeroacoustic source mechanism in high speed, high Reynolds number jets. 

By rearranging the Navier-Stokes equations, Lighthill \citep{Lighthill1952} was able to transform the governing equations for fluid dynamics into an inhomogeneous convected wave equation. 
In this acoustic analogy, the source term comprises Reynolds stress, shear stress, and density fluctuation terms (commonly referred to as \emph{Lighthill's stress tensor}).
As this formulation is exact (aside from the assumption of a constant sound speed), complete knowledge of the source field will yield an exact solution for the acoustic far field.
In practical applications (e.g. high-speed, turbulent jets) however, the full source field cannot be measured using current experimental capabilities nor simulated with sufficient fidelity, thereby requiring certain simplifications.
While great progress has been made in the field of aeroacoustics, both experimentally \citep{Tam1996, Viswanathan2006, Tam2008} as well as theoretically \citep{Cabana2008}, understanding of jet noise sources and their radiation mechanisms remains incomplete \citep{Jordan2008}.

Where to go from here? \ldots