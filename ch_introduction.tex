\chapter{Introduction}
\label{introduction}
\section{Motivation}
The advent of the turbojet engine led to a transformation in both commercial and military aviation, allowing for much faster flight than previously possible with propellor-driven aircraft. 
However, the increased thrust of turbojets has come at great cost; significant acoustic radiation is generated by the rotating components (compressor, turbine, fan), by the combustion process, and ultimately by the free jet itself. 
On the commercial side, the escalating number of flights, encroachment of urban and residential areas near airports, and tightening of environmental regulations have combined to force airports to institute curfews, surcharges and flight path restrictions to combat noise pollution. 
For the military, hearing damage inflicted on nearby personnel (particularly flight deck crew on aircraft carriers) has necessitated the implementation of noise reduction concepts on tactical aircraft.
During takeoff and landing, when acoustic radiation is most problematic to ground crew and  surrounding urban and residential areas, the dominant noise source of the jet engine is the aeroacoustic radiation generated by the high velocity engine exhaust.
This has spurred extensive research, spanning over six decades, into the acoustic source mechanism in high speed, high Reynolds number jets. 

While progress has been made in the field of aeroacoustics, both experimentally \citep{Tam1996, Viswanathan2006, Tam2008} as well as theoretically \citep{Cabana2008}, understanding of jet noise sources and their radiation mechanisms remains incomplete \citep{Jordan2008}.
This is due to the large number of interrelated parameters (e.g. Reynolds number, temperature ratio, acoustic Mach number, nozzle geometry, et cetera) as well as the large disparity in the associated length and time scales of the turbulent phenomena and the radiated noise.
Simulations of controlled free shear layers have suggested that there is significant potential for noise reduction, on the order of 11 dB in some cases [Wei 2006].
However, these simulations relied on non-physically defined actuation (that is, forcing was applied over a defined region by arbitrary energy, momentum, and body force terms), and a physical interpretation of the optimum forcing parameters was not immediately clear to the researchers.
Current noise-mitigation technologies for free jets have largely been applied in an adhoc fashion, due to our incomplete understanding of the aeroacoustic sources.
Fully realizing this maximum noise reduction potential will require a much more detailed understanding of the mechanism (or mechanisms) by which free jets radiate to the far-field.

It is generally agreed that the dominant noise sources are related to the large-scale turbulent structures present in the mixing layer of the jet. 
What remains to be determined is what aspects of the large-scale structure evolution and interactions are relevant to the noise generation process. 
Theoretical models of spatially- and temporally-modulated wavepackets have shown great promise in replicating the observed characteristics of the dominant far-field noise [Cavalieri/Jordan?]. 
However, direct experimental data linking this structure evolution to the acoustic emission is still lacking. 
It is on this vein that the current work is focused.
Until recently, experimental data acquisition techniques have been unable to capture the flow physics with enough fidelity (lacking in either spatial or temporal resolution) in order to accurately model the large-scale structures and aeroacoustic sources.
By combining contemporary data acquisition methods (free-field microphones and non-time-resolved particle image velocimetry) with novel post-processing algorithms this work aims to directly link the relevant vortex dynamics of the large-scale structures to the acoustic emission events, and in the process identify a simplified aeroacoustic source mechanism. 
	
\section{Background}
\subsection{Flow Control}
Controlling the development of the jet plume, and hence controlling the rate of mixing or intensity and characteristics of the emitted acoustic radiation, is a long running goal of the aeroacoustic community.
Passive, permanent modifications to the nozzle have been shown to be quite adept at this task; some examples of these include tabs [citations] and chevrons [citations]. 
These work to generate counter-rotating streamwise vortices in the developing shear layer, which serve to substantially increase mixing between the core and coflow in the near-nozzle region and ultimately retard the growth of large-scale axisymmetric structures [citation].

Unfortunately, these passive modifications have associated penalties to the engine performance, in terms of added weight or reduced thrust.
Due to the passive nature of the flow modification, these performance penalties are in effect over the entire duration of the flight regardless of whether or not the noise reduction is needed. 
To improve engine efficiency, active control techniques are desired, since they can be activated when needed, such as during takeoff and landing, and deactivated when unneeded, such as after a commercial airliner reaches cruising altitude. 
Active control techniques, which seek to manipulate instabilities in the jet shear layer, have been extensively studied in low-speed, low-Reynolds number jets, the most common of which is acoustic drivers [citations]. 
However, as the speed and Reynolds number of the jet is increased (to match those in practical applications), so too does the required bandwidth and energy of the active drivers. 
Hence, acoustic or magneto-hydrodynamic drivers lose control authority in these regimes, and more powerful actuators are required.

The last decade has seen a rapid growth in the development of plasma actuators for use in high-speed flow control; though as of yet they have not progressed past the experimental phase.
Localized arc filament plasma actuators (LAFPAs) are one such class of plasma actuator, which were developed by a collaboration between the Gas Dynamics and Turbulence Laboratory (GDTL) and the Non-Equilibrium Thermodynamics Laboratory (NETL) at the Ohio State University.
LAFPAs can provide the high-amplitude and high-frequency excitation required for control of high Mach number and high Reynolds number jets [citations]. 
GDTL has used these actuators for noise mitigation and flow control in Mach 0.9 [citations], Mach 1.3 [citations] and Mach 1.65 [citations] jets (both heated and unheated). 
A review of the development of LAFPAs and their use in flow control and fluid phenomena research in high speed, high Reynolds number jets can be found in Samimy et al. [citation]. 
More recently, the diagnostic potential of LAFPAs for understanding jet flow phenomena has been explored. Excitation of instabilities in the flow by LAFPAs results in a definitive spatio-temporal origin to which resulting phenomena can be referenced. 
The absolute temporal reference afforded by LAFPA excitation provides researchers the ability to investigate the growth, saturation, and decay of structures with high fidelity. 
An example of their diagnostic potential can be found in the work of Kearney-Fischer et al. [citation], which investigated Mach wave radiation from heated, high Mach number jets using schlieren imaging phase-locked to LAFPAs. 

Unlike their passive counterparts (such as tabs or chevrons), or some other potential active flow control technologies (such as fluidic chevrons), LAFPAs control the shear layer development indirectly by exciting naturally occurring instabilities.
The sharp velocity gradient in the jet shear layer gives rise to the inviscid Kelvin-Helmholtz instability [citation].



TO DO:
	\begin{itemize}
		\item shear layer / jet instabilities
		\item large-scale structures / energy cascade / wavepackets
	\end{itemize}
\subsection{Components of Jet Noise}
A simplified model of the noise generation process in stationary free jets can be found in Fig. [INSERT FIGURE]. 
This model is based off of the work of Tam et al \citep{Tam1996, Tam2008}, who observed that the far-field spectra could be represented as a combination of two similarity spectra based on polar angle of the observer, regardless of jet Mach number or temperature.
At observer angles close to the jet downstream axis, the spectra exhibited a clearly defined spectral peak (\emph{F}-spectrum), whereas at sideline or upstream angles the spectra were broadband (\emph{G}-spectrum).
From this observation the two-component acoustic source model was born: fine-scale turbulence, dominant in the near-nozzle region, is responsible for the omni-directional acoustic radiation that dominates the sideline and upstream polar angles. On the other hand, the large-scale turbulent structures which exist further downstream produce the superdirective radiation that is readily apparent at aft polar angles. 
 
Additional noise source mechanisms have been identified for supersonic jets. In imperfectly expanded jets, shock cells are produced in the jet. As turbulent structures pass through these waves, the sharp pressure gradients cause them to emit acoustic radiation. 
This is observed directly in the far-field as a broad-band amplification at high frequencies, referred to simply as broad-band shock-associated noise (BBSAN). 
In stationary or subsonic airframes this radiation can generate a feedback loop, whereby the noise travels upstream to the nozzle exit, excites the initial shear layer, and produces new structures at the same frequency.
A high-amplitude, narrow-band tone (screech noise) is the end result of this feedback loop.
Lastly, supersonically-convecting (relative to the ambient) large-scale structures (which exist in supersonic and heated jets) produce high-amplitude, strongly-directional acoustic radiation towards aft angles.
This Mach wave radiation can be explained by a wavy-wall analogy (Tam again?).
In the present work, the jet is unheated and subsonic; as such these noise sources are not present and therefore neglected throughout the rest of this work.

TO DO:
	\begin{itemize}
		\item	Create simplified figure of free jet showing high frequency versus low frequency radiation (based off of Tam 2008)
		\item 	Explain Tam's "two-component" source model, with additional experimental evidence?
	\end{itemize}

\subsection{Acoustic Source Models}
By rearranging the Navier-Stokes equations, Lighthill \citep{Lighthill1952} was able to transform the governing equations for fluid dynamics into an inhomogeneous convected wave equation. 
In this acoustic analogy, the source term comprises Reynolds stress, shear stress, and density fluctuation terms (commonly referred to as \emph{Lighthill's stress tensor}).
As this formulation is exact (aside from the assumption of a constant sound speed), complete knowledge of the source field will yield an exact solution for the acoustic far field.
In practical applications (e.g. high-speed, turbulent jets) however, the full source field cannot be measured using current experimental capabilities nor simulated with sufficient fidelity, thereby requiring certain simplifications.
