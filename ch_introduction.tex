\chapter{Introduction}
\label{introduction}
\section{Motivation}
The advent of the turbojet engine led to a transformation in both commercial and military aviation, allowing for much faster flight than previously possible with propellor-driven aircraft. 
However, the increased thrust of turbojets has come at great cost; significant acoustic radiation is generated by the rotating components (compressor, turbine, fan), by the combustion process, and ultimately by the free jet itself. 
On the commercial side, the escalating number of flights, encroachment of urban and residential areas near airports, and tightening of environmental regulations have combined to force airports to institute curfews, surcharges and flight path restrictions to combat noise pollution. 
For the military, hearing damage inflicted on nearby personnel (particularly on aircraft carriers) has necessitated the implementation of noise reduction concepts on tactical aircraft.
During takeoff and landing, when acoustic radiation is most problematic to ground crew and  surrounding urban and residential areas, the dominant noise source of the jet engine is the aeroacoustic radiation generated by the high velocity engine exhaust.
This has spurred extensive research, spanning over six decades, into the acoustic source mechanism in high speed, high Reynolds number jets. 

While progress has been made in the field of aeroacoustics, both experimentally \citep{Tam1996, Viswanathan2006, Tam2008} as well as theoretically \citep{Cabana2008}, understanding of jet noise sources and their radiation mechanisms remains incomplete \citep{Jordan2008}.
This is due to the large number of interrelated parameters (e.g. Reynolds number, temperature ratio, acoustic Mach number, nozzle geometry, et cetera) as well as the large disparity in the associated length and time scales of the turbulent phenomena and the radiated noise.
Simulations of controlled free shear layers have suggested that there is significant potential for noise reduction, on the order of 11 dB in some cases [Wei 2006].
However, these simulations relied on non-physically defined actuation (that is, forcing was applied over a defined region by arbitrary energy, momentum, and body force terms), and a physical interpretation of the forcing parameters was not immediately clear to the researchers.
Current noise-mitigation technologies for free jets have largely been applied in an adhoc fashion, due to our incomplete understanding of the aeroacoustic sources.
Fully realizing this maximum noise reduction potential will require a much more detailed understanding of the mechanism (or mechanisms) by which free jets radiate to the far-field.

It is generally agreed that the dominant noise sources are related to the large-scale turbulent structures present in the mixing layer of the jet. 

TO DO:
	\begin{itemize}
		\item	Explain purpose of this particular work
	\end{itemize}
	
\section{Background}
\subsection{Components of Jet Noise}
A simplified model of the noise generation process in stationary free jets can be found in Fig. [INSERT FIGURE]. 
This model is based off of the work of Tam et al \citep{Tam1996, Tam2008}, who observed that the far-field spectra could be represented as a combination of two similarity spectra based on polar angle of the observer, regardless of jet Mach number or temperature.
At observer angles close to the jet downstream axis, the spectra exhibited a clearly defined spectral peak (\emph{F}-spectrum), whereas at sideline or upstream angles the spectra were broadband (\emph{G}-spectrum).

Additional noise source mechanisms have been identified for supersonic jets. In imperfectly expanded jets, shock cells are produced in the jet. As turbulent structures pass through these waves, the sharp pressure gradients cause them to emit acoustic radiation. 
This is observed directly in the far-field as a broad-band amplification at high frequencies, referred to simply as broad-band shock-associated noise (BBSAN). 
In stationary or subsonic airframes this radiation can generate a feedback loop, whereby the noise travels upstream to the nozzle exit, excites the initial shear layer, and produces new structures at the same frequency.
A high-amplitude, narrow-band tone (screech noise) is the end result of this feedback loop.
Lastly, supersonically-convecting (relative to the ambient) large-scale structures (which exist in supersonic and heated jets) produce high-amplitude, strongly-directional acoustic radiation towards aft angles.
This Mach wave radiation can be explained by a wavy-wall analogy (Tam again?).
In the present work, the jet is unheated and subsonic; as such these noise sources are not present and therefore neglected throughout the rest of this work.

TO DO:
	\begin{itemize}
		\item	Create simplified figure of free jet showing high frequency versus low frequency radiation (based off of Tam 2008)
		\item 	Explain Tam's "two-component" source model, with additional experimental evidence?
	\end{itemize}

\subsection{The Acoustic Analogy}
By rearranging the Navier-Stokes equations, Lighthill \citep{Lighthill1952} was able to transform the governing equations for fluid dynamics into an inhomogeneous convected wave equation. 
In this acoustic analogy, the source term comprises Reynolds stress, shear stress, and density fluctuation terms (commonly referred to as \emph{Lighthill's stress tensor}).
As this formulation is exact (aside from the assumption of a constant sound speed), complete knowledge of the source field will yield an exact solution for the acoustic far field.
In practical applications (e.g. high-speed, turbulent jets) however, the full source field cannot be measured using current experimental capabilities nor simulated with sufficient fidelity, thereby requiring certain simplifications.
\subsection{Flow Control}